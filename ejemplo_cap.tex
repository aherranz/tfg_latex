\chapter{Ejemplo de capítulo}
[Esto es un ejemplo de capítulo]

En este capítulo incluimos varias secciones y en cada una de ellas se presenta un recurso básico de LaTeX. Hay muchos más recursos.

\section{Tipografía}

Puedes cambiar ciertas características del tipo de letra: \textrm{texto en ``roman font''}, \textbf{texto en negrita}, \emph{texto enfatizad}, \textit{texto en itálica}, \texttt{texto en teletype}, \textsc{Texto En Small Caps}.

Por supuesto puedes combinar: \textbf{\textit{texto itálica en negrita}}

\section{Capítulo, secciones subsecciones, etc.}

Como ya has visto, en LaTeX jamás pones un número de referencia, simplemente declaras dónde empieza un capítuo, una sección o una subsección. Como la siguiente.

\subsection{Título de una subsección}

Dentro pondremos una subsubsección.

\subsubsection{Título de una subsubsección}

Dentro pondremos un ``párrafo'' con título:

\paragraph{Título de un párrafo}
Y el texto del párrafo sigue.

\subsection{Seccions sin números}

\section*{Título de sección sin número}

\subsection*{Título de subsección sin número}

\section{Listas de cosas}

Puedes usar listas no numeradas:
\begin{itemize}
\item Cosa uno
\item Cosa dos
\item Cosa tres
\item Cosa cuatro
\end{itemize}

O listas numeradas:
\begin{enumerate}
\item Cosa uno
\item Cosa dos
\item Cosa tres
\item Cosa cuatro
\end{enumerate}

\section{Figuras}

Puedes poner cosas dentro de una figura. Por ejemplo la
figura~\ref{fig:escudo}. LaTeX siempre intenta colocar las figuras en
el ``mejor'' sitio.

\begin{figure}[h]
  \centering
  \includegraphics[width=0.33\linewidth]{portada/etsiinf}
  \caption{El escudo de la ETSIINF}
  \label{fig:escudo}
\end{figure}

\section{Matemáticas}

LaTeX está muy preparado para escribir fórmulas matemáticas con variables como $x$ en expresiones como esta en línea: \(\int_{a}^{b} x^2 \,dx\) o en un párrafo centrado a parte:

\begin{displaymath}
  \oint_V f(s) \,ds
\end{displaymath}

\section{Espaciados verticales}

Trata siempre de evitar los comando \verb|\vspace|, \verb|\newpage|, \verb|\clearpage|, \verb|\\|, etc.

\section{Citas bibligráficas}

Las citas bibliográficas se incluyen de esta forma: puede encontrar
las recomendaciones para realizar el TFG en
\cite{recomendaciones}. Para añadir nuevas citas deberás poner las
entradas en el fichero \url{*.bib} y luego puedes referenciarla.

Esta es la cita bibliográfica de un libro \cite{ec}.

\section{Ejemplo de ``por hacer'' (\emph{todonotes})}

Por supuesto puedes poner TODOS:\todo{como este en el margen}.

\todo[inline]{O como este ``inline''}

\section{Ejemplo de inclusión de código fuente}

A continuación se muestra un listado de código usando el paquete listings. En él se puede ver la función \lstinline{main()} de un programa en C que hace un \emph{hello world}.
\begin{lstlisting}[language=c]
#include <stdio.h>
// A simple Hello World
int main(){
  printf("Hello World!\n");
  return 0;
}
\end{lstlisting}

\section{Lorem Ipsum}
Lo que sigue es un lorem ipsum como ejemplo de lo que sería una sección relativametne larga.

\lipsum


%%% Local Variables:
%%% mode: latex
%%% TeX-master: "tfg_main"
%%% TeX-PDF-mode: t
%%% ispell-local-dictionary: "castellano"
%%% End:
